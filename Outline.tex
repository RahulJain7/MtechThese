\documentclass[12pt]{article}
\topmargin -0.75in
\textheight 9.25in
\textwidth 6.25in
\oddsidemargin 0in
\evensidemargin 0in

\usepackage{enumitem}
\usepackage{verbatim}
\renewcommand{\topfraction}{1.0}
\renewcommand{\bottomfraction}{1.0}
\renewcommand{\textfraction}{0}
\renewcommand{\floatpagefraction}{1.0}
\title{Outline for Mtech Stage II Project Report}
\author{}
\date{}
\begin{document}
\maketitle
\section{The report will be written based on this outline}
\begin{enumerate}
\item Chemical Process simulator:
\begin{enumerate}[label*=\arabic*.]
\item What is it.
\item The three major components of a process simulator
\begin{enumerate}[label*=\arabic*.]
\item Compound Database.
\item Thermodynamics
\item Unit Operation Modules.
\end{enumerate}
\item Application of process simulators in education and industry. 
\end{enumerate}

\item Requirement: Need of an open source process simulator
\begin{enumerate}[label*=\arabic*.]
\item Commercial process simulator are very expensive, if used for educational purposes. Due to which most of the chemical engineering institutes exclude process simulation course from their curriculum
\item Commercial process simulators are unaffordable for small and medium scale industries. Therefore most of them are poorly designed, which reduces the efficiency of the plant.
\end{enumerate}

\item DWSIM: An open source steady state process simulator
\begin{enumerate}[label*=\arabic*.]
\item Follows sequential modular approach
\item Intensive compound database.
\item Good build in library of Unit operation and excellent thermodynamics.
\item Contains extra utilities like graph plotting and optimization.
\item Provides standalone thermodynamic library which can be used externally.
\item Drawbacks: 
\begin{enumerate}[label*=\arabic*.]
\item Not suitable for design problems.
\item Most of the small scale industries consists of batch processes which needs dynamic simulation for which DWSIM cannot be used.
\end{enumerate}
\end{enumerate}


\item Openmodelica: An open source simulation environment for solving engineering models.
\begin{enumerate}[label*=\arabic*.]
 \item Follows equation oriented approach.
 \item Contains good built in solvers.
 \item Excellent debugging features
 \item Based on object oriented approach which has lots of advantages like code reuse and encapsulation.
 \item Connectors: an advanced concepts for connecting object.
 \item Provides Excellent GUI where each object can be directly connected with other through connectors. Therefore reduces manual coding.
 \item Drawbacks: As it is a general software, there is no built in thermodynamics and unit operations. 
 \end{enumerate}

\item Importing the Thermodynamic engine of DWSIM in Openmodelica. 
\begin{enumerate}[label*=\arabic*.]
\item The two approaches used for the integration.
\begin{enumerate}[label*=\arabic*.]
\item Python-C Api approach.
\item Client-Server approach (sockets). 
\end{enumerate}
\item Comparison of the two approaches. Based on the time taken to simulate the same example.
\end{enumerate}

\item Development of a built in Thermodynamic engine in Openmodelica itself.
\begin{enumerate}[label*=\arabic*.]
\item Requirement for a built in thermodynamic engine.
\item Development of the three components of a thermodynamic engine.
\begin{enumerate}[label*=\arabic*.]
\item Development of Compound Database.
\item Development of Thermodynamic Functions.
\item Development of Thermodynamic Packages.
\end{enumerate}
\item Comparison with the earlier approaches.
\end{enumerate}

\item Solved examples in Openmodelica by using built in thermodynamics.
\begin{enumerate}[label*=\arabic*.]
\item Generating Bubble point and Dew point curves for ethanol water system by using NRTL and UNIQUAC thermodynamic models.
\item Steady state flash model capable of doing design calculations as well
\item Dynamic flash model.
\item Bubble batch distillation of Methanol and water from Ingham.
\item Batch distillation with temperature controller for reflux.(part of PHD thesis by Efstathios S Iliopoulos)
\end{enumerate}

\item Future Work
\begin{enumerate}[label*=\arabic*.]
\item Developing steady state and dynamic models of all the unit operation in Openmodelica.
\item Validating the above models by comparing the simulated results with a commercial simulator.
\item Documentation of the developed thermodynamics and unit modules.

\end{enumerate}
\end{enumerate}
\end{document}